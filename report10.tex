\documentclass[a4paper,11pt,twoside]{article}
%\documentclass[a4paper,11pt,twoside,se]{article}

\usepackage{UmUStudentReport}
\usepackage{verbatim}   % Multi-line comments using \begin{comment}
\usepackage{courier}    % Nicer fonts are used. (not necessary)
\usepackage{pslatex}    % Also nicer fonts. (not necessary)
\usepackage[pdftex]{graphicx}   % allows including pdf figures
\usepackage{listings}
\usepackage{pgf-umlcd}
\usepackage{blindtext}
\usepackage{enumitem}
\usepackage{scrextend}
%\usepackage{lmodern}   % Optional fonts. (not necessary)
%\usepackage{tabularx}
%\usepackage{microtype} % Provides some typographic improvements over default settings
%\usepackage{placeins}  % For aligning images with \FloatBarrier
%\usepackage{booktabs}  % For nice-looking tables
%\usepackage{titlesec}  % More granular control of sections.

% DOCUMENT INFO
% =============
\department{Department of Applied Physics and Electronics}
\coursename{Linux as Development Environment 7.5 ECTS}
\coursecode{5EL142 HT-16}
\title{Assignment 7 - Various Tools: groff, diff, patch, Awk, sed }
\author{Lorenz Gerber, 20161202-2033 ({\tt{lorenzottogerber@gmail.com}})}
\date{2017-05-30}
%\revisiondate{2016-01-18}
\instructor{Sven Rönnbäck, John Berge, Björne A Lindberg}


% DOCUMENT SETTINGS
% =================
\bibliographystyle{plain}
%\bibliographystyle{ieee}
\pagestyle{fancy}
\raggedbottom
\setcounter{secnumdepth}{2}
\setcounter{tocdepth}{2}
%\graphicspath{{images/}}   %Path for images

\usepackage{float}
\floatstyle{ruled}
\newfloat{listing}{thp}{lop}
\floatname{listing}{Listing}



% DEFINES
% =======
%\newcommand{\mycommand}{<latex code>}

% DOCUMENT
% ========
\begin{document}
\lstset{language=C}
\maketitle
\thispagestyle{empty}
\newpage
%\tableofcontents
%\thispagestyle{empty}
%\newpage

\clearpage
\pagenumbering{arabic}

\section{Introduciton}
Short report with a reflection about the work. What went well what was difficult. Name some limitations of the chosen solution and/or propose some improvements to the provided scripts.

\section{Timetrack}
Timetracker was implemented using the \verb+case+ statement. The tracker information is stored in a file called time.log. The script handles the cases when there is either a tracker running or not for all three arguments.
The commmand \verb?date +%s? is used to obtain the seconds since 1970-01-01 00:00:00 UTC. Then \verb?typeset -1? is used to convert and store the value as an integer variable for later arithemtics.

\section{Checkuser}
Checkuser is a script that checks if a specific user is logged into the system. The user name is given either as command line argument or is queried interactively. \verb+read+ is used to obtain the user input. \verb+awk+ is used to extract usernames of all logged in users from the output of command \verb+who+. The whole process of who, awk and test is lined up with pipe commands.

\section{Check5D}
Check5D is a script that checks the \verb+make+ build script of exercise 5D. The script tests five different operations:
\begin{itemize}
\item \verb+make all+ executes without errors
\item executables are created
\item executable execute and return correct result
\item \verb+make install+ works
\item \verb+make uninstall+ works
\item \verb+make clean+ works
\end{itemize}

\section{Test Suite for Electronics Applicaiton}
Below for a short pseudo code on how the test suite was implemented.

\begin{verbatim}
checkUsage()
createTempFile
run electrotest as background process and divert output into tempfile
get PID from background process
sleep for a short time
if backprocess still runs
    outOfTime or doesNotCatchException
    kill background process
else
    wait PID
    store exit status
    check exit status
fi 
\end{verbatim}

\addcontentsline{toc}{section}{\refname}
%\bibliography{references}

\end{document}
