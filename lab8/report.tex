\documentclass[a4paper,11pt,twoside]{article}
%\documentclass[a4paper,11pt,twoside,se]{article}

\usepackage{UmUStudentReport}
\usepackage{verbatim}   % Multi-line comments using \begin{comment}
\usepackage{courier}    % Nicer fonts are used. (not necessary)
\usepackage{pslatex}    % Also nicer fonts. (not necessary)
\usepackage[pdftex]{graphicx}   % allows including pdf figures
\usepackage{listings}
\usepackage{pgf-umlcd}
%\usepackage{lmodern}   % Optional fonts. (not necessary)
%\usepackage{tabularx}
%\usepackage{microtype} % Provides some typographic improvements over default settings
%\usepackage{placeins}  % For aligning images with \FloatBarrier
%\usepackage{booktabs}  % For nice-looking tables
%\usepackage{titlesec}  % More granular control of sections.

% DOCUMENT INFO
% =============
\department{Department of Applied Physics and Electronics}
\coursename{Linux as Development Environment 7.5 ECTS}
\coursecode{5EL142 HT-16}
\title{Assignment 8}
\author{Lorenz Gerber, 20161202-2033 ({\tt{lorenzottogerber@gmail.com}})}
\date{2016-01-02}
%\revisiondate{2016-01-18}
\instructor{Björne A. Lindberg}


% DOCUMENT SETTINGS
% =================
\bibliographystyle{plain}
%\bibliographystyle{ieee}
\pagestyle{fancy}
\raggedbottom
\setcounter{secnumdepth}{2}
\setcounter{tocdepth}{2}
%\graphicspath{{images/}}   %Path for images

\usepackage{float}
\floatstyle{ruled}
\newfloat{listing}{thp}{lop}
\floatname{listing}{Listing}



% DEFINES
% =======
%\newcommand{\mycommand}{<latex code>}

% DOCUMENT
% ========
\begin{document}
\lstset{language=C}
\maketitle
\thispagestyle{empty}
\newpage
%\tableofcontents
%\thispagestyle{empty}
%\newpage

\clearpage
\pagenumbering{arabic}

\section{GDB} 
After download and untaring the archieve, I first compiled the program with \textit{make} and run it. The screen output of the program showed two mathematical wrong terms. Second, I quickly inspected the source files, starting with \textit{main.c}, then \textit{gdblab.c}. Obviously, the output is generated by reading values from an array that is filled with values using a \textit{for-loop}.

\subsection{Changing Makefile}
To debug, I added the \textit{g} flag to the compile and link commands as shown below:
\begin{verbatim}
CC = gcc
LIBFLAG = -L. -lgdb -Wl,-rpath,.

all:	program

program:	main.c libgdb.so
		$(CC) -g -o program main.c $(LIBFLAG)

libgdb.so:	lib/gdblab.c lib/gdblab.h 
		$(CC) -g -c -fPIC lib/gdblab.c
		$(CC) -g -shared -fPIC -o libgdb.so gdblab.o
\end {verbatim}


\subsection{Debugging}
The debugger was started with \textit{gdb ./program}. First the command \textit{list} was used to get a quick view on the source code. Here it was seen that the code of interest is not within \textit{main.c} but in the \textit{test} function from the dynamically loaded library \textit{gdblab.c}. Hence a break point for the \textit{test} function was set by \textit{break test}. Then the program was started with \textit{run}. After hitting the breakpoint, \textit{list} was used again to get an overview of the source code. Now \textit{display test->buffert1} and \textit{display test->buffert2} was used to get updated values for the respective variables in every step. Finally, using \textit{next}, respectively the shortcut \textit{n} was used to step through the rest of the code. After every step, the above mentioned variables/datastructures are presented on the screen.

\subsection{Fixing the code}
In the above described debugging session, it became obvious that the problem was a too small allocated array, \textit{test->buffert1}. It had size 16. In each for loop from 0 to 16 the index value was assigned to the corresponding array index. However, 0 to 16 are 17 values. Hence the last value, which is needed in the following print statement, is written in the zero index position of \textit{text->buffert2}. There it is overwritten later. Hence, the first print statement that accesses test->buffert1[16] actually accesses test->buffert2[0] with the value 45. This bug can be fixed by increasing the test->buffert1 array to size 17. The second print statement accesses the wrong array (\textit{test->buffert1} instead of \textit{test->buffert2}). The corrected source code is shown below:

\begin{verbatim}
#include <stdio.h>
#include <stdlib.h>
#include <sys/types.h>
#include <signal.h>

#include "gdblab.h"

int test()
{
	int i;
	typedef struct{
		int buffert1[17];
		int buffert2[16];
	}theTest;

	theTest *test = malloc(sizeof(theTest));

	int j=2;

	for (i=0; i<=16; i++)
	{
		test->buffert1[i] = i;
	}

	test->buffert2[0] = 45;
	test->buffert2[1] = 5;

	printf("16 + 14 = %i\n", test->buffert1[16] + test->buffert1[14]);
	printf("45 + 5 = %i\n", test->buffert2[0] + test->buffert2[1]);
	free(test);

	//test->buffert2[0] = 5;
	//printf("16 + 14 = %i\n", test->buffert1[16] + test->buffert1[14]);

	//kill(getpid(),SIGSEGV);

return 0;
}

\end{verbatim}

\section{Trace}
The program was downloaded and made executable by \textit{chmod +x errorprog}. Then the program was run from the command line. On exit, \textit{echo \$?} returns 255. According to \textit{www.tldp.org}, `Advanced Bash Scripting', `Appendix E. Exit Codes with Special Meanings', 255 is the `out of range' exit code as exit takes only values between 0 to 255.

Running \textit{strace -o output.txt ./errorprog} produces the following output:

\begin{verbatim}
execve("./errorprog", ["./errorprog"], [/* 74 vars */]) = 0
brk(NULL)                               = 0x1ce0000
access("/etc/ld.so.nohwcap", F_OK)      = -1 ENOENT (No such file or directory)
mmap(NULL, 8192, PROT_READ|PROT_WRITE, MAP_PRIVATE|MAP_ANONYMOUS, -1, 0) = 0x7f85f120a000
access("/etc/ld.so.preload", R_OK)      = -1 ENOENT (No such file or directory)
open("/etc/ld.so.cache", O_RDONLY|O_CLOEXEC) = 3
fstat(3, {st_mode=S_IFREG|0644, st_size=130544, ...}) = 0
mmap(NULL, 130544, PROT_READ, MAP_PRIVATE, 3, 0) = 0x7f85f11ea000
close(3)                                = 0
access("/etc/ld.so.nohwcap", F_OK)      = -1 ENOENT (No such file or directory)
open("/lib/x86_64-linux-gnu/libc.so.6", O_RDONLY|O_CLOEXEC) = 3
read(3, "\177ELF\2\1\1\3\0\0\0\0\0\0\0\0\3\0>\0\1\0\0\0P\t\2\0\0\0\0\0"..., 832) = 832
fstat(3, {st_mode=S_IFREG|0755, st_size=1864888, ...}) = 0
mmap(NULL, 3967392, PROT_READ|PROT_EXEC, MAP_PRIVATE|MAP_DENYWRITE, 3, 0) = 0x7f85f0c1e000
mprotect(0x7f85f0ddd000, 2097152, PROT_NONE) = 0
mmap(0x7f85f0fdd000, 24576, PROT_READ|PROT_WRITE, MAP_PRIVATE|MAP_FIXED|MAP_DENYWRITE, 3, 0x1bf000) = 0x7f85f0fdd000
mmap(0x7f85f0fe3000, 14752, PROT_READ|PROT_WRITE, MAP_PRIVATE|MAP_FIXED|MAP_ANONYMOUS, -1, 0) = 0x7f85f0fe3000
close(3)                                = 0
mmap(NULL, 4096, PROT_READ|PROT_WRITE, MAP_PRIVATE|MAP_ANONYMOUS, -1, 0) = 0x7f85f11e9000
mmap(NULL, 4096, PROT_READ|PROT_WRITE, MAP_PRIVATE|MAP_ANONYMOUS, -1, 0) = 0x7f85f11e8000
mmap(NULL, 4096, PROT_READ|PROT_WRITE, MAP_PRIVATE|MAP_ANONYMOUS, -1, 0) = 0x7f85f11e7000
arch_prctl(ARCH_SET_FS, 0x7f85f11e8700) = 0
mprotect(0x7f85f0fdd000, 16384, PROT_READ) = 0
mprotect(0x600000, 4096, PROT_READ)     = 0
mprotect(0x7f85f120c000, 4096, PROT_READ) = 0
munmap(0x7f85f11ea000, 130544)          = 0
open("data.txt", O_RDONLY)              = -1 ENOENT (No such file or directory)
read(-1, 0x7ffeb604d8b0, 250)           = -1 EBADF (Bad file descriptor)
fstat(1, {st_mode=S_IFCHR|0620, st_rdev=makedev(136, 4), ...}) = 0
brk(NULL)                               = 0x1ce0000
brk(0x1d01000)                          = 0x1d01000
write(1, "Read \n", 6)                  = 6
close(-1)                               = -1 EBADF (Bad file descriptor)
exit_group(-1)                          = ?
+++ exited with 255 +++
\end{verbatim}

It can be seen that \textit{errorprog} tries to open a file name \textit{data.txt}. Which initially results in a `ENOENT', `No such file or directory' error. The error value returned to bash however comes from the last command, \textit{close} which tries to close the file again. The program has obviously no proper error handling in place as the first error on \textit{open} does not lead to disregarding the \textit{close} command which results in an `EABADF', `Bad file descriptor' error. By standard definition Unix C programs/functions return -1 on error. As -1 is not in the range 0 to 255, an exit code of 255 results in bash.

Now a file name \textit{data.txt} with some text in it was created and the program was invoked again with \textit{strace -o output.txt ./errorprog} producing the following \textit{output.txt}:

\begin{verbatim}
execve("./errorprog", ["./errorprog"], [/* 74 vars */]) = 0
brk(NULL)                               = 0x23a8000
access("/etc/ld.so.nohwcap", F_OK)      = -1 ENOENT (No such file or directory)
mmap(NULL, 8192, PROT_READ|PROT_WRITE, MAP_PRIVATE|MAP_ANONYMOUS, -1, 0) = 0x7f92487f3000
access("/etc/ld.so.preload", R_OK)      = -1 ENOENT (No such file or directory)
open("/etc/ld.so.cache", O_RDONLY|O_CLOEXEC) = 3
fstat(3, {st_mode=S_IFREG|0644, st_size=130544, ...}) = 0
mmap(NULL, 130544, PROT_READ, MAP_PRIVATE, 3, 0) = 0x7f92487d3000
close(3)                                = 0
access("/etc/ld.so.nohwcap", F_OK)      = -1 ENOENT (No such file or directory)
open("/lib/x86_64-linux-gnu/libc.so.6", O_RDONLY|O_CLOEXEC) = 3
read(3, "\177ELF\2\1\1\3\0\0\0\0\0\0\0\0\3\0>\0\1\0\0\0P\t\2\0\0\0\0\0"..., 832) = 832
fstat(3, {st_mode=S_IFREG|0755, st_size=1864888, ...}) = 0
mmap(NULL, 3967392, PROT_READ|PROT_EXEC, MAP_PRIVATE|MAP_DENYWRITE, 3, 0) = 0x7f9248207000
mprotect(0x7f92483c6000, 2097152, PROT_NONE) = 0
mmap(0x7f92485c6000, 24576, PROT_READ|PROT_WRITE, MAP_PRIVATE|MAP_FIXED|MAP_DENYWRITE, 3, 0x1bf000) = 0x7f92485c6000
mmap(0x7f92485cc000, 14752, PROT_READ|PROT_WRITE, MAP_PRIVATE|MAP_FIXED|MAP_ANONYMOUS, -1, 0) = 0x7f92485cc000
close(3)                                = 0
mmap(NULL, 4096, PROT_READ|PROT_WRITE, MAP_PRIVATE|MAP_ANONYMOUS, -1, 0) = 0x7f92487d2000
mmap(NULL, 4096, PROT_READ|PROT_WRITE, MAP_PRIVATE|MAP_ANONYMOUS, -1, 0) = 0x7f92487d1000
mmap(NULL, 4096, PROT_READ|PROT_WRITE, MAP_PRIVATE|MAP_ANONYMOUS, -1, 0) = 0x7f92487d0000
arch_prctl(ARCH_SET_FS, 0x7f92487d1700) = 0
mprotect(0x7f92485c6000, 16384, PROT_READ) = 0
mprotect(0x600000, 4096, PROT_READ)     = 0
mprotect(0x7f92487f5000, 4096, PROT_READ) = 0
munmap(0x7f92487d3000, 130544)          = 0
open("data.txt", O_RDONLY)              = 3
read(3, "123\n", 250)                   = 4
fstat(1, {st_mode=S_IFCHR|0620, st_rdev=makedev(136, 4), ...}) = 0
brk(NULL)                               = 0x23a8000
brk(0x23c9000)                          = 0x23c9000
write(1, "Read 123\n", 9)               = 9
write(1, "\n", 1)                       = 1
close(3)                                = 0
exit_group(0)                           = ?
+++ exited with 0 +++

\end{verbatim}

Now in the trace it can be seen that \textit{open(``data.txt'', O\_RDONLY)} returns 3. Looking up the man pages for the system call \textit{open} shows that this function returns the assigned file descriptor on success. Hence the assigned file descriptor in this case is `3'. In the end, the program exits with exit code 0, the standard Unix C exit code for success.

\addcontentsline{toc}{section}{\refname}
\bibliography{references}

\end{document}
