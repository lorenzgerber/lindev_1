\documentclass[a4paper,11pt,twoside]{article}
%\documentclass[a4paper,11pt,twoside,se]{article}

\usepackage{UmUStudentReport}
\usepackage{verbatim}   % Multi-line comments using \begin{comment}
\usepackage{courier}    % Nicer fonts are used. (not necessary)
\usepackage{pslatex}    % Also nicer fonts. (not necessary)
\usepackage[pdftex]{graphicx}   % allows including pdf figures
\usepackage{listings}
\usepackage{pgf-umlcd}
\usepackage{blindtext}
\usepackage{enumitem}
\usepackage{scrextend}
%\usepackage{lmodern}   % Optional fonts. (not necessary)
%\usepackage{tabularx}
%\usepackage{microtype} % Provides some typographic improvements over default settings
%\usepackage{placeins}  % For aligning images with \FloatBarrier
%\usepackage{booktabs}  % For nice-looking tables
%\usepackage{titlesec}  % More granular control of sections.

% DOCUMENT INFO
% =============
\department{Department of Applied Physics and Electronics}
\coursename{Linux as Development Environment 7.5 ECTS}
\coursecode{5EL142 HT-16}
\title{Assignment 2}
\author{Lorenz Gerber, 20161202-2033 ({\tt{lorenzottogerber@gmail.com}})}
\date{2016-08-20}
%\revisiondate{2016-01-18}
\instructor{Sven Rönnbäck}


% DOCUMENT SETTINGS
% =================
\bibliographystyle{plain}
%\bibliographystyle{ieee}
\pagestyle{fancy}
\raggedbottom
\setcounter{secnumdepth}{2}
\setcounter{tocdepth}{2}
%\graphicspath{{images/}}   %Path for images

\usepackage{float}
\floatstyle{ruled}
\newfloat{listing}{thp}{lop}
\floatname{listing}{Listing}



% DEFINES
% =======
%\newcommand{\mycommand}{<latex code>}

% DOCUMENT
% ========
\begin{document}
\lstset{language=C}
\maketitle
\thispagestyle{empty}
\newpage
%\tableofcontents
%\thispagestyle{empty}
%\newpage

\clearpage
\pagenumbering{arabic}

\section{Describing Basic Linux Commands} 

\begin{labeling}{\textbf{man -k} \textit{expression}}
\item [\textbf{man} \textit{keyword}] display manual pages for a specific command 
\item [\textbf{man -k} \textit{expression}] display the short descriptions that contain a regex match to \textit{expression}
\item [man -f \textit{keyword}] display the short descriptions that contain the \textit{keyword} 
\end{labeling}

\subsubsection{\textbf{info} \textit{topic}- display info document}
textbf{infor -} \textit{topic} \\
\textbf{cp} - copy a file or directory \\
\textbf{mv} - move a file or directory \\
\textbf{mkdir} - make a directory 
\textbf{rmdir} - remove an empty directory
\textbf{rm} - remove files
\textbf{find} - search files
\textbf{cd} - change directory
\textbf{pwd} - present working directory
\textbf{df} - disk free
\textbf{ps} - processes
\textbf{du} - disk usage
\textbf{tar} - archieving multiple files into one
\textbf{seq} - generate a sequence of numbers
\textbf{whoami} - display the userid of the current user
\textbf{users} - user names of currently logged on users at present host
\textbf{who} - shows the currently logged on users
\textbf{whereis} - find binary, source and man pages for a specified command
\textbf{cat} - concatenating files and write to standard out
\textbf{tee} - allows writing standard input to multiple out 
\textbf{more} - display text page wise
\textbf{less} - display multi page text with scrolling
\textbf{uniq} - match repeated lines, either report or discard
\textbf{tail} - output the end of files
\textbf{echo} - display a line of text
\textbf{which} - find the path to a command
\textbf{wget} - download files over the network
\textbf{cut} - print selected parts of lines
\textbf{grep} - print lines that match a given pattern
\textbf{sort} - sorting lines
\textbf{wc} - word count

\section{User Access}

\section{Users and Groups}


\section{Filesystem}
/boot
/etc
/sbin
/bin
/usr
/var
/dev
/home

\section{Pipes}
a. ls | sort
b. uniq | wc
c. users| uniq | sort
d. seq | tee test1 test2

\section{streams}


\addcontentsline{toc}{section}{\refname}
\bibliography{references}

\end{document}
