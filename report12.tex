\documentclass[a4paper,11pt,twoside]{article}
%\documentclass[a4paper,11pt,twoside,se]{article}

\usepackage{UmUStudentReport}
\usepackage{verbatim}   % Multi-line comments using \begin{comment}
\usepackage{courier}    % Nicer fonts are used. (not necessary)
\usepackage{pslatex}    % Also nicer fonts. (not necessary)
\usepackage[pdftex]{graphicx}   % allows including pdf figures
\usepackage{listings}
\usepackage{pgf-umlcd}
\usepackage{blindtext}
\usepackage{enumitem}
\usepackage{scrextend}
%\usepackage{lmodern}   % Optional fonts. (not necessary)
%\usepackage{tabularx}
%\usepackage{microtype} % Provides some typographic improvements over default settings
%\usepackage{placeins}  % For aligning images with \FloatBarrier
%\usepackage{booktabs}  % For nice-looking tables
%\usepackage{titlesec}  % More granular control of sections.

% DOCUMENT INFO
% =============
\department{Department of Applied Physics and Electronics}
\coursename{Linux as Development Environment 7.5 ECTS}
\coursecode{5EL142 HT-16}
\title{Assignment 12 - Graphical Development Environments - Eclipse }
\author{Lorenz Gerber, 20161202-2033 ({\tt{lorenzottogerber@gmail.com}})}
\date{2017-08-13}
%\revisiondate{2016-01-18}
\instructor{Sven Rönnbäck, John Berge, Björne A Lindberg}


% DOCUMENT SETTINGS
% =================
\bibliographystyle{plain}
%\bibliographystyle{ieee}
\pagestyle{fancy}
\raggedbottom
\setcounter{secnumdepth}{2}
\setcounter{tocdepth}{2}
%\graphicspath{{images/}}   %Path for images

\usepackage{float}
\floatstyle{ruled}
\newfloat{listing}{thp}{lop}
\floatname{listing}{Listing}



% DEFINES
% =======
%\newcommand{\mycommand}{<latex code>}

% DOCUMENT
% ========
\begin{document}
\lstset{language=C}
\maketitle
\thispagestyle{empty}
\newpage
%\tableofcontents
%\thispagestyle{empty}
%\newpage

\clearpage
\pagenumbering{arabic}

\section{Introduction} The aim of this assignment was to learn about graphical
development environments such as Eclipse. To achieve certain exposure to
Eclipse, two practical exercises had to be conducted. First, the GTK+
electrotest application from lab 11 had to be setup for development in Eclipse
and extended with a button to run the given test library linumtest.h. Eventual
bugs had to be found and corrected. For the second part, a C++/GTK+ `Mandelbrot'
application had to be set up in Eclipse and extended with zoom in/out
functionality.

The general idea about `Graphical Ingegrated Development Environments' is to
provide language tailored, advanced tooling for developing larger software
projects. Some exampels of such tools are code completion, graphical debugging,
graphical library managment, dependency checking, versioning etc. Usually, such
graphical IDE's such as eclipse simply intgerate available console tools with a
graphical possibility to choose the options.

\section{Method}
Here it was chosen to install `eclipse neon' using the `Oomph' installer. The `Ooomph'
installer is a separate piece of software that provisions the installation of
`eclipse' development environments. After startup of `oomph', `Eclipse CDT' the
eclipse version for C/C++ development is chosen, then the whole IDE is installed
automatically.

\subsection{electrotestgtk}
Setup of the electrotestgtk project for development was conducted according to the
instructions given on the course homepage in the following order:
\begin{enumerate}
  \item `Setting up projects in Eclipse'
  \item `Configuration of Eclipse Projects for GTK+-2.0'
  \item `Installation and Use of own Libraries in Eclipse'
\end{enumerate}

The first step, setting up a project was followed as described. For the GTK+-2.0
configuration, some additional steps were needed that the librarires were found
correctly. Namely, `/usr/include/gtk-2.0' and `/usr/include/glib.2.0' had to be
added in `Properties' -> `C/C++ Build' -> `Settings' ->  `Cross GCC compiler' ->
`includes'. Similar, `/usr/share/glib-2.0' was added in `Properties' -> `C/C++
Build' ->  `Settings' -> `Cross GCC Linker'-> `Libraries'.

Installation of own libraries worked fine according the instructions. Prior to
installation, new library versions with debugger flag (-g) set were generated.

After including the linumtest library, it was found that the component library
did not adhere to the given function names, hence, this was corrected. Then
the button and a callback function was added to the the electrotestgtk application.

Debugging was conducted by graphically setting breakpoints, starting the debug
process by clicking the respective button and then using `step over' and/or
`step into' functionality while inspecting the values of variables.

\subsection{Mandelbrot}
Setting up of the Mandelbrot project was correspondingly to the `electrotestgtk'
project except for choosing a `C++' template. All the `include' and `Libarary'
settings were done for the `g++' build toolchain.

Initially the structure of the program was analyzed by debugger runs to step
through the program. The `time\_handler' function was used to find and test
suitable functions for implementing the zoom functionality in regard to mouse
pointer positioning. Briefly, the pointer position was obtained as absolut
position on the visible screen (`gdk\_display\_get\_pointer'). Then the `Gdk\_window'
which in this has the same size as the main application window without the
window decorations, was obtained using `gtk\_widget\_get\_window'. Finally, the
size of the Gdk\_window was determined by `gdk\_window\_get\_size'. Knowing the
fixed size of the `GdkImage' drawing area, the absolut position of the mouse
pointer when within the GdkImage could be calculated.

For the zoom functionality it was established that a GdkImage length and height
of each 800px corresponds the value `2' in the fractal calculation algorithm. The
input parameter for the fractal algorithm are the x/y value of the mid-image, which
then corresponds to 400px/400px in GdkImage screen measures.

Zoom functionality was implemented as a single left click for zoom-in
and single right-click for zoom-out. During testing, it was found that zoom-out
should not change the center of the image but rather just zoom-out.

Four different compiler flags (O0, O1, O2, O3, Os) where tested for execution
speed of the compiled code aswell as for binary size. Settings within
Eclipse IDE where used to set the compiler flags. As a metric for code
execution, the `time.h' library was imported and `clock()' was used to measure
execution time for the first fractal calculated on application startup.

\section{Results}
\subsection{electrotestgtk}
As mentioned earlier, prior to debugging, the component library had to be
recompiled with the correct function names. Then the linumtest program was
run. It run without crash, however the test results indicated an error
in the component library. Now the debugger was used to step trough the code
and analyze the it's function. For lab 6, I was in charge of another library,
so I was not fully aware of how the component library works. The test failed
in the `bTestComponent' function. On stepping through, it was found that the
second call to `e\_resistance' with the input parameter `100' expects a return
value of `2', however e\_resistance returns 1, hence the test fails. According
to my understanding of how E12 resistances are calculated, 100 is part of the
series, hence the result should indeed be 1. To amend the problem, the test
function was modified to expect `1' as return value. Now all tests passed
and the GTK button turns green.

\subsection{mandelbrot}
The zoom functionality allows zooming in and out of the fractal by simply
clicking the area of interest for zoom-in. Zoom-out keeps the current location.
The zoom function makes use of the `zoom' input argument of the `bCalc\_Fractal'
function.

\begin{table}[]
\centering
\caption{Contains the benchmark values for calculating the first fractal using
different compiler optimization settings. Further, size of binary is also indicated.}
\label{tab:benchmark}
\begin{tabular}{lll}
flag & mean (sec) & binary size (K) \\ \hline
O0   & 3.21       & 502   \\
O1   & 1.01       & 518   \\
O2   & 0.84       & 521   \\
O3   & 0.84       & 522   \\
Os   & 1.38       & 514
\end{tabular}
\end{table}

\section{Discussion}
Generally, setting up the grarphical IDE `Eclipse' was straight forward and without
any problems. I used `Eclipse' already for Lab 10, altough with a external makefile
setup. Hence in this lab, the only new thing was to set up the libraries and
includes for linking and compilation. It took a while in the beginning to understand
which setting reflects which change on the auto-generated makefile. When understood,
a graphical IDE allows for very quick and simple setup. Eclipse has since some time
back also the feature (Oomph) that all settings can be written into configuration files so
that for example a company could deploy a complicated development environment for
a new employee within minutes. For me the striking advantage of an IDE such as Eclipse
over a normal text editor or even a more advnaced setup with Emacs is the 'mouse over'
feature that generate pop-ups, for example over macros, or functions with additional
information. Also the code completion features are really neat.

\subsection{electrotestgtk}
Setting up the application from lab 11 in Eclipse was without problems. Using
Eclipse for developing GTK+ applications seems to be a good choice as it simplifies
for getting quick information about all the macros. Also graphical debugging feels
like a serious productivity boos to me. I find this interesting as most of the tools
and functionality available in Eclipse are basically the same as on the commandline.



\subsection{mandelbrot}
The run-time benchmarks showed that compilator optimization could speed up
the tested calcuation three fold. From optimization level O2 to O3, no significant
improvement was measured. The size of binary files showed that speed optimized
ones where slightly larger in sequence of optimization level. The space optimized
compilation was smaller than the all speed optimized, however still larger than
the non optimized O0 version.

\section{Conclusion}
I used Eclipse a lot at work with Java, however only a few times so far with
small one file C exercises. I enjoyed learning how to set up libraries in Eclipse
and I was amazed how much code completion and mouse-over info boxes can help in
C development - features that I so far only took for granted in Java development
environments.

I did only see the document `Hints on implementing Mandelbrot Fractal' after I
already finished my version. Of course that is my fault, but if you want to improve
something in the documentation, I think you could mention in the `Övning 12
Grafiska Utvecklingsmiljöer' document that there are additional documents related
to the `mandelbrot' implementation. Then again, the lab was on a fair level
even without the hints. I got some version running quite quick, but it took a
while until I realized that I had to apply a scaling factor. Without it, the
zooming got with every step of zoom more and more sensitive on choosing the
zoom spot. As long as one stayed close to the image center it worked. Finally,
after having found the but, zooming in and out the Mandelbrot kept me playing
happily for at least half an hour (smile). That's what I call a rewarding
exercise, really nice.


\addcontentsline{toc}{section}{\refname}
%\bibliography{references}

\end{document}
