\documentclass[a4paper,11pt,twoside]{article}
%\documentclass[a4paper,11pt,twoside,se]{article}

\usepackage{UmUStudentReport}
\usepackage{verbatim}   % Multi-line comments using \begin{comment}
\usepackage{courier}    % Nicer fonts are used. (not necessary)
\usepackage{pslatex}    % Also nicer fonts. (not necessary)
\usepackage[pdftex]{graphicx}   % allows including pdf figures
\usepackage{listings}
\usepackage{pgf-umlcd}
\usepackage{blindtext}
\usepackage{enumitem}
\usepackage{scrextend}
%\usepackage{lmodern}   % Optional fonts. (not necessary)
%\usepackage{tabularx}
%\usepackage{microtype} % Provides some typographic improvements over default settings
%\usepackage{placeins}  % For aligning images with \FloatBarrier
%\usepackage{booktabs}  % For nice-looking tables
%\usepackage{titlesec}  % More granular control of sections.

% DOCUMENT INFO
% =============
\department{Department of Applied Physics and Electronics}
\coursename{Linux as Development Environment 7.5 ECTS}
\coursecode{5EL142 HT-16}
\title{Assignment 12 - Graphical Development Environments - Eclipse }
\author{Lorenz Gerber, 20161202-2033 ({\tt{lorenzottogerber@gmail.com}})}
\date{2017-08-08}
%\revisiondate{2016-01-18}
\instructor{Sven Rönnbäck, John Berge, Björne A Lindberg}


% DOCUMENT SETTINGS
% =================
\bibliographystyle{plain}
%\bibliographystyle{ieee}
\pagestyle{fancy}
\raggedbottom
\setcounter{secnumdepth}{2}
\setcounter{tocdepth}{2}
%\graphicspath{{images/}}   %Path for images

\usepackage{float}
\floatstyle{ruled}
\newfloat{listing}{thp}{lop}
\floatname{listing}{Listing}



% DEFINES
% =======
%\newcommand{\mycommand}{<latex code>}

% DOCUMENT
% ========
\begin{document}
\lstset{language=C}
\maketitle
\thispagestyle{empty}
\newpage
%\tableofcontents
%\thispagestyle{empty}
%\newpage

\clearpage
\pagenumbering{arabic}

\section{Introduction} The aim of this assignment was to learn about graphical
development environments such as Eclipse. To achieve certain exposure to
Eclipse, two practical exercises had to be conducted. First, the GTK+
electrotest application from lab 11 had to be setup for development in Eclipse
and extended with a button to run the given test library linumtest.h. Eventual
bugs had to be found and corrected. For the second part, a C++/GTK+ `Mandelbrot'
application had to be set up in Eclipse and extended with zoom in/out
functionality.

The general idea about `Graphical Ingegrated Development Environments' is to
provide language tailored, advanced tooling for developing larger software
projects. Some exampels of such tools are code completion, graphical debugging,
graphical library managment, dependency checking, versioning etc. Usually, such
graphical IDE's such as eclipse simply intgerate available console tools with a
graphical possibility to choose the options.

\section{Method}
Here it was chosen to install `eclipse neon' using the `Oomph' installer. The `Ooomph'
installer is a separate piece of software that provisions the installation of
`eclipse' development environments. After startup of `oomph', `Eclipse CDT' the
eclipse version for C/C++ development is chosen, then the whole IDE is installed
automatically.

\subsection{electrotestgtk}
Setup of the electrotestgtk project for development was conducted according to the
instructions given on the course homepage in the following order:
\begin{enumerate}
  \item `Setting up projects in Eclipse'
  \item `Configuration of Eclipse Projects for GTK+-2.0'
  \item `Installation and Use of own Libraries in Eclipse'
\end{enumerate}

The first step, setting up a project was followed as described. For the GTK+-2.0
configuration, some additional steps were needed that the librarires were found
correctly. Namely, `/usr/include/gtk-2.0' and `/usr/include/glib.2.0' had to be
added in `Properties' -> `C/C++ Build' -> `Settings' ->  `Cross GCC compiler' ->
`includes'. Similar, `/usr/share/glib-2.0' was added in `Properties' -> `C/C++
Build' ->  `Settings' -> `Cross GCC Linker'-> `Libraries'.

Installation of own libraries worked fine according the instructions. Prior to
installation, new library versions with debugger flag (-g) set were generated.

After including the linumtest library, it was found that the component library
did not adhere to the given function names, hence, this was corrected. Then
the button and a callback function was added to the the electrotestgtk application.

Debugging was conducted by graphically setting breakpoints, starting the debug
process by clicking the respective button and then using `step over' and/or
`step into' functionality while inspecting the values of variables.

\subsection{Mandelbrot}
Setting up of the Mandelbrot project was correspondingly to the `electrotestgtk'
project except for choosing a `C++' template. All the `include' and `Libarary'
settings were done for the `g++' build toolchain.

Initially the structure of the program was analyzed by debugger runs to step
through the program. The `time_handler' function was used to find and test the
suitable functions for implementing the zoom functionality.

Zoom functionality was finally implemented as a single left click for zoom-in
and single right-click for zoom-out. 



- compiler flags for optimizing code
\section{Results}
\subsection{electrotestgtk}
As mentioned earlier, prior to debugging, the component library had to be
recompiled with the correct function names. Then the linumtest program was
run. It run without crash, however the test results indicated an error
in the component library. Now the debugger was used to step trough the code
and analyze the it's function. For lab 6, I was in charge of another library,
so I was not fully aware of how the component library works. The test failed
in the `bTestComponent' function. On stepping through, it was found that the
second call to `e\_resistance' with the input parameter `100' expects a return
value of `2', however e\_resistance returns 1, hence the test fails. According
to my understanding of how E12 resistances are calculated, 100 is part of the
series, hence the result should indeed be 1. To amend the problem, the test
function was modified to expect `1' as return value. Now all tests passed
and the GTK button turns green.

Result part II

\section{Discussion}
It was identifed that the `time_handler' function was
used mostly for `debugging output' to the console, however, it contained a
number of usefull commands to be used for the zoom function.


\section{Conclusion}



\addcontentsline{toc}{section}{\refname}
%\bibliography{references}

\end{document}
