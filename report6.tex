\documentclass[a4paper,11pt,twoside]{article}
%\documentclass[a4paper,11pt,twoside,se]{article}

\usepackage{UmUStudentReport}
\usepackage{verbatim}   % Multi-line comments using \begin{comment}
\usepackage{courier}    % Nicer fonts are used. (not necessary)
\usepackage{pslatex}    % Also nicer fonts. (not necessary)
\usepackage[pdftex]{graphicx}   % allows including pdf figures
\usepackage{listings}
\usepackage{pgf-umlcd}
\usepackage{blindtext}
\usepackage{enumitem}
\usepackage{scrextend}
%\usepackage{lmodern}   % Optional fonts. (not necessary)
%\usepackage{tabularx}
%\usepackage{microtype} % Provides some typographic improvements over default settings
%\usepackage{placeins}  % For aligning images with \FloatBarrier
%\usepackage{booktabs}  % For nice-looking tables
%\usepackage{titlesec}  % More granular control of sections.

% DOCUMENT INFO
% =============
\department{Department of Applied Physics and Electronics}
\coursename{Linux as Development Environment 7.5 ECTS}
\coursecode{5EL142 HT-16}
\title{Assignment 6 - Libraries}
\author{LORENZ GERBER, 20161202-2033 ({\tt{lorenzottogerber@gmail.com}})}
\date{2016-11-03}
%\revisiondate{2016-01-18}
\instructor{Sven Rönnbäck, Björne A Lindberg}


% DOCUMENT SETTINGS
% =================
\bibliographystyle{plain}
%\bibliographystyle{ieee}
\pagestyle{fancy}
\raggedbottom
\setcounter{secnumdepth}{2}
\setcounter{tocdepth}{2}
%\graphicspath{{images/}}   %Path for images

\usepackage{float}
\floatstyle{ruled}
\newfloat{listing}{thp}{lop}
\floatname{listing}{Listing}



% DEFINES
% =======
%\newcommand{\mycommand}{<latex code>}

% DOCUMENT
% ========
\begin{document}
\lstset{language=C}
\maketitle
\thispagestyle{empty}
\newpage
%\tableofcontents
%\thispagestyle{empty}
%\newpage

\clearpage
\pagenumbering{arabic}

\section{Library `libpower'}
\subsection{Function and Usage}
I implemented the library `libpower' and small program to test it's functionality. The usage of the test program is straight forward as it does not need any command line arguments. After calling the program (which we here can call `test'), a short text-based user menu is shown to access the two functions included in the library. The two functions are: 1. calculate power in a circuit from voltage and resistance and 2. calculate power in a circuit from voltage and and current. After choosing the function by entering either `R' or `I' and pressing the `enter' key, the program will query for the respective values and then calculate power. The result is printed to stdout.  

\subsection{Algorithm}
In the case of `libpower' there was not much of an algorithm to implement. The functions could be written directly as arithmetic expressions and needed also no special library.  

\subsection{Building}
Building the library and the test program was done in three steps:
\begin{verbatim}
gcc -Wall -std=c11 -pedantic -c -fPIC libpower.c
gcc -shared -o libpower.so libpower.o
gcc -Wall -std=c11 -pedantic -c test.c
gcc -o test test.o -L. -lpower -Wl,-rpath,.
\end{verbatim}
First the library is compiled


\subsubsection{Used Compiler / Linker Flags}

\section{`electrotest' and `makefile'}
\subsection{Group Members}
\subsection{Usage of the Dynamic Libraries}
\subsection{Buidling}

\section{Collaborative Development}
\subsection{Account on Current Collaborative Work}
\subsection{Large Collaborative Projects}

\addcontentsline{toc}{section}{\refname}
%\bibliography{references}

\end{document}
