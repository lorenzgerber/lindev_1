\documentclass[a4paper,11pt,twoside]{article}
%\documentclass[a4paper,11pt,twoside,se]{article}

\usepackage{UmUStudentReport}
\usepackage{verbatim}   % Multi-line comments using \begin{comment}
\usepackage{courier}    % Nicer fonts are used. (not necessary)
\usepackage{pslatex}    % Also nicer fonts. (not necessary)
\usepackage[pdftex]{graphicx}   % allows including pdf figures
\usepackage{listings}
\usepackage{pgf-umlcd}
\usepackage{blindtext}
\usepackage{enumitem}
\usepackage{scrextend}
%\usepackage{lmodern}   % Optional fonts. (not necessary)
%\usepackage{tabularx}
%\usepackage{microtype} % Provides some typographic improvements over default settings
%\usepackage{placeins}  % For aligning images with \FloatBarrier
%\usepackage{booktabs}  % For nice-looking tables
%\usepackage{titlesec}  % More granular control of sections.

% DOCUMENT INFO
% =============
\department{Department of Applied Physics and Electronics}
\coursename{Linux as Development Environment 7.5 ECTS}
\coursecode{5EL142 HT-16}
\title{Assignment 13 - Debian Package Handling}
\author{Lorenz Gerber, 20161202-2033 ({\tt{lorenzottogerber@gmail.com}})}
\date{2017-08-13}
%\revisiondate{2016-01-18}
\instructor{Sven Rönnbäck, John Berge, Björne A Lindberg}


% DOCUMENT SETTINGS
% =================
\bibliographystyle{plain}
%\bibliographystyle{ieee}
\pagestyle{fancy}
\raggedbottom
\setcounter{secnumdepth}{2}
\setcounter{tocdepth}{2}
%\graphicspath{{images/}}   %Path for images

\usepackage{float}
\floatstyle{ruled}
\newfloat{listing}{thp}{lop}
\floatname{listing}{Listing}



% DEFINES
% =======
%\newcommand{\mycommand}{<latex code>}

% DOCUMENT
% ========
\begin{document}
\lstset{language=C}
\maketitle
\thispagestyle{empty}
\newpage
%\tableofcontents
%\thispagestyle{empty}
%\newpage

\clearpage
\pagenumbering{arabic}

\section{Introduction}
This lab was about creating Debian packages from the application written for
lab 6 and lab 11 respectively.

Litterature study on various web resources showed that there are basically two
different approaches: Starting with a tar.gz source archieve of the so called
`upstream software' to be packaged for debian, or create directly a binary
package. The former will ultimately also result in a binary package. It's the
more generic way where the source is first compiled for the respective platform.
As the lab instructions and forum entries suggested that both is viable, it was
decided to produce directly binary packages.

\section{Method}
\subsection{Basics}
First a suitable binary has to be obtained or created. For normal programs, the
requirements are not very specific. Here it was however attempted to also build
a proper shared library package. For this, the binary had to be rebuild according
to the specifications (e.g. using -soname flag)\cite{debShlibs}.

Then, the directory tree has to be set up and the binary file(s) copied into it.
Further, the required configuration files have to be created and copied into the
directory tree.

Then the package can be build.

Finally, the result should be checked using the debian package linter `lintian'.
Usually, the file modes have to be adjusted. It was found that the last two steps,
`building' and `linting' were iterated until a satisfactiory result was obtained.

\subsection{Tools}
For producing a binary deb package, almost no tools besides a text editor were
needed:
\begin{enumerate}
\item gcc build system to first produce the binaries
\item dpkg-deb --build: to create the actual package
\item fakeroot: used with dpgk-deb, to create the package as root user
\item lintian: linter tool to check quality of the deb package
\end{enumerate}

For building source packages, a number of other tools that help to set up the
initial directory tree and template files would be available.

\section{Detailed Description}
\subsection{libelectro1}
First the the libraries where rebuild using the -soname flag \cite{tldpShlibs}.
\begin{verbatim}
gcc -fPIC -c -Wall libresistance.c
gcc -fPIC -c -Wall libpower.c
gcc -fPIC -c -Wall libcomponent.c
gcc -shared -Wl,-soname,libelectro.so.1 \
    -o libelectro.so.1.0.1 libresistance.o libpower.o libcomponent.o -lc
\end{verbatim}

This resulted in a shared library: \verb+libelectro.so-1.0.1+. A symbolic
link \verb+libelectro.so.1+ was also created:

\begin{verbatim}
  ln -s libelectro.so-1.0.1 libelectro.so.1
\end{verbatim}

Then a new directory tree was build:
\begin{verbatim}
  mkdir -p ./debian/usr/lib
  mkdir -p ./debian/DEBIAN
\end{verbatim}
The library and the symbolic link where copied into \verb+./debian/usr/lib+. Then
in \verb+./debian/DEBIAN+ three new files where created: \verb+control+,
\verb+triggers+ and \verb+shlibs+.

The \verb+control+ file was edited as follows \cite{controlfiles}:
\begin{verbatim}
Package: libelectro1
Version: 1.0-1
Section: libs
Priority: optional
Architecture: amd64
Depends: libc6 (>= 2.2.1)
Maintainer: L. Gerber <lorenz.gerber@provement.se>
Description: library with functions to calculate
 e12 replacement resistance values.
\end{verbatim}

The \verb+triggers+ file contains only one line to trigger
ldconfig \cite{debShlibs}:
\begin{verbatim}
activate-noawait ldconfig
\end{verbatim}

The shlibs file contains the following line \cite[8.6.4.2.]{debShlibs}:
\begin{verbatim}
libelectro 1 libelectro1
\end{verbatim}
Then the binary package is created:
\begin{verbatim}
fakeroot dpkg-deb --build libelectro-1.0/
mv libelectro-1.0.deb libelectro.so-1.0-1_amd64.deb
\end{verbatim}

And finally the quality of the produced package is checked using:
\begin{verbatim}
lintian libelectro.so-1.0-1_amd64.deb
\end{verbatim}

This resulted in the follwing output:
\begin{verbatim}
E: libelectro1: unstripped-binary-or-object usr/lib/libelectro.so.1.0.1
E: libelectro1: debian-changelog-file-missing
E: libelectro1: no-copyright-file
\end{verbatim}

The following copyright file was added in
\verb+libelectro-1.0/usr/share/doc/libelectro1/+:
\begin{verbatim}
Format: https://www.debian.org/doc/packaging-manuals/copyright-format/1.0/
Upstream-Name: electrotest
Source: https://github.com/lorenzgerber/electrotest

Files: *
Copyright: 2017 Lorenz Gerber <lorenz.gerber@provement.se>
License: GPL-2+
 This program is free software; you can redistribute it
 and/or modify it under the terms of the GNU General Public
 License as published by the Free Software Foundation; either
 version 2 of the License, or (at your option) any later
 version.
 .
 This program is distributed in the hope that it will be
 useful, but WITHOUT ANY WARRANTY; without even the implied
 warranty of MERCHANTABILITY or FITNESS FOR A PARTICULAR
 PURPOSE.  See the GNU General Public License for more
 details.
 .
 You should have received a copy of the GNU General Public
 License along with this package; if not, write to the Free
 Software Foundation, Inc., 51 Franklin St, Fifth Floor,
 Boston, MA  02110-1301 USA
 .
 On Debian systems, the full text of the GNU General Public
 License version 2 can be found in the file
 `/usr/share/common-licenses/GPL-2'.
\end{verbatim}

The changelog file had to be gzipped with the parameters \verb+-n -9+. Lintian
is very peculiar about the format of the changelog file. The `ITM close bug'
issue still shown in `linitan'  can not be mended as it is a mechanism used
when a package is developed to be included in the debian distribution: Lintian
recognizes that the changelog only contains one entry, hence it assumes a new
package. New packages have to be announced by filing a bug-report in the ITM
system. Then the bug number has to be mentioned in the first entry to automatically
close it. The following is the example text for the changelog file to be situated
in \verb+usr/share/doc/electrolib1/+:
\begin{verbatim}
libelectro1 (1.0-1) UNRELEASED; urgency=low

  * Initial release.
  Not intended to be included in Debian, hence closes no initial bug.

 -- Lorenz Gerber <lorenz.gerber@provement.se>  Fri, 18 Aug 2017 8:08:00 +0000
\end{verbatim}
Now `lintian' indicates that the package is clean. It can now be installed using:
\begin{verbatim}
sudo dpkg -i libelectro.so-1.0-1_amd64.deb
\end{verbatim}
and uninstalled using:
\begin{verbatim}
sudo dpkg -r libelectro1
\end{verbatim}



\addcontentsline{toc}{section}{\refname}
\bibliography{references}

\end{document}
