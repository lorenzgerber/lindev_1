\documentclass[a4paper,11pt,twoside]{article}
%\documentclass[a4paper,11pt,twoside,se]{article}

\usepackage{UmUStudentReport}
\usepackage{verbatim}   % Multi-line comments using \begin{comment}
\usepackage{courier}    % Nicer fonts are used. (not necessary)
\usepackage{pslatex}    % Also nicer fonts. (not necessary)
\usepackage[pdftex]{graphicx}   % allows including pdf figures
\usepackage{listings}
\usepackage{pgf-umlcd}
\usepackage{blindtext}
\usepackage{enumitem}
\usepackage{scrextend}
%\usepackage{lmodern}   % Optional fonts. (not necessary)
%\usepackage{tabularx}
%\usepackage{microtype} % Provides some typographic improvements over default settings
%\usepackage{placeins}  % For aligning images with \FloatBarrier
%\usepackage{booktabs}  % For nice-looking tables
%\usepackage{titlesec}  % More granular control of sections.

% DOCUMENT INFO
% =============
\department{Department of Applied Physics and Electronics}
\coursename{Linux as Development Environment 7.5 ECTS}
\coursecode{5EL142 HT-16}
\title{Assignment 7 - Various Tools: groff, diff, patch, Awk, sed }
\author{LORENZ GERBER, 20161202-2033 ({\tt{lorenzottogerber@gmail.com}})}
\date{2016-11-26}
%\revisiondate{2016-01-18}
\instructor{Sven Rönnbäck, John Berge, Björne A Lindberg}


% DOCUMENT SETTINGS
% =================
\bibliographystyle{plain}
%\bibliographystyle{ieee}
\pagestyle{fancy}
\raggedbottom
\setcounter{secnumdepth}{2}
\setcounter{tocdepth}{2}
%\graphicspath{{images/}}   %Path for images

\usepackage{float}
\floatstyle{ruled}
\newfloat{listing}{thp}{lop}
\floatname{listing}{Listing}



% DEFINES
% =======
%\newcommand{\mycommand}{<latex code>}

% DOCUMENT
% ========
\begin{document}
\lstset{language=C}
\maketitle
\thispagestyle{empty}
\newpage
%\tableofcontents
%\thispagestyle{empty}
%\newpage

\clearpage
\pagenumbering{arabic}

\section{Groff}
Below follows the source for generating the man pages. The parsed postscript files are attached as files. They were generated by \verb+man -t command > command.ps+. 
\subsection{Source man page for \textit{calc\_power\_i}}
\begin{verbatim}
.\" Manpage for calc_power_i.
.\" Contact lorenzottogerberls@gmail.com to correct errors or typos.
.TH CALC_POWER_I 1 "13 September 2016"
.SH NAME
calc_power_i \- library function to calculate power from voltage and current
.SH SYNOPSIS
.B #include <libpower.h>
.sp
.B float calc_power_i(float volt, float current);

.SH DESCRIPTION
Calculates
.I power in
.I watt
from
.I voltage and
.I current
according
.B volt * current.
.SH RETURN VALUE
function returns the power in watt as a float
.SH CONFORMING TO
C11
.SH BUGS
currently no bugs known
.SH SEE ALSO
.BR calc_power_r(1)
.SH AUTHOR
Lorenz Gerber (lorenzottogerber@gmail.com)
\end{verbatim}

\subsection{Source man page for \textit{calc\_power\_r}}

\begin{verbatim}
.\" Manpage for calc_power_r.
.\" Contact lorenzottogerberls@gmail.com to correct errors or typos.
.TH CALC_POWER_R 1 "13 September 2016"
.SH NAME
calc_power_i \- library function to calculate power from voltage and resistance
.SH SYNOPSIS
.B #include <libpower.h>
.sp
.B float calc_power_r(float volt, float resistance);
.SH DESCRIPTION
Calculates power in
.I watt
from
.I voltage
and
.I resistance
according
.B (voltage)^2 / resistance
in
.I ohm.
.SH RETURN VALUE
function returns the power in watt as a float
.SH CONFORMING TO
C11
.SH BUGS
currently no bugs known
.SH SEE ALSO
.BR calc_power_i(1)
.SH AUTHOR
Lorenz Gerber (lorenzottogerber@gmail.com)


\end{verbatim}


\section{Diff and Patch}
\subsection{Common flags for \textit{diff} and \textit{patch}}
\begin{itemize}
\item \textbf{diff -r} recursive, is used to climb recursively down in subdirectories and compare the files.
\item \textbf{diff -u} the unified display format. It is a variation of the context format which omits reduanant context lines hence it is very compact. 
\item \textbf{diff -N} Does treat inexisting files as empty, hence diff will show the whole existing file as difference
\item \textbf{patch -p\textit{number} / --strip=\textit{number}} strips parts of the path from files contained in the patch. By default, only the filename is returned. \verb+-p0+ returns the whole path, \verb+-p1+ strips one slash, \verb+-p2+ strips two slashes etc.  

\end{itemize}


\subsection{Diff and patch applied to a directory tree (2b assignment) }
A directory directory structe \textit{liborig/subdir/} with \textit{file1.txt}, \textit{file2.txt} in \textit{liborig/} and \textit{file3.txt}, \textit{file4.txt} in \textit{liborig/subdir/} was constructed then copied to \textit{libwork/subdir} still containing the same files. Then \textit{file2.txt} and \textit{file4.txt} in the \textit{libwork} directory were modified.

To check the difference visually on the screen, \verb+diff -ruN liborig/ libwork/+ was run. After verifying the changes, a patch file was written by adding the redirection \verb+> patch.txt+ to the prior diff command. The \textit{patch.txt} file contains now the patches in the respective files with the full path from the directory where diff was called. Hence to apply the patches to the \textit{liborig/} and it's subdirectory, we need to enter \textit{liborig/} and apply the patch however with chopping off the first directory level (\textit{libwork}) by using the \textit{-p1} flag: \verb+patch -p1 < ../patch.txt+. \textit{../} is used to access the actual patch file which is now one level higher in the directory structure.


\section{Awk and sed}
\subsection {Print user name and comments from \textit{/etc/passwd}}
\verb+awk -F : '{print $1, $5}' /etc/passwd+
\subsection {Use sed to replace swedish umlauts}
A text from a swedish newspaper was copy/pasted into a file svensktext.txt. Then the following command was applied: \\
\verb+sed 's/[äÄ]/ae/g ; s/[åÅ]/aa/g ; s/[öÖ]/oe/g' svensktext.txt > modified.txt+

\addcontentsline{toc}{section}{\refname}
%\bibliography{references}

\end{document}
